\documentclass[11pt,article]{memoir}
\usepackage{multirow}
\usepackage{agd-assignment}
\usepackage{fancybox}
\usepackage{afterpage}
\usepackage{graphicx}
\usepackage{standalone}
\usepackage[yyyymmdd,hhmmss]{datetime}


\usepackage{listings}
\lstset{
    frame=single,
    breaklines=true,
    postbreak=\raisebox{0ex}[0ex][0ex]{\ensuremath{\color{red}\hookrightarrow\space}}
}
\let\footruleskip\undefined
\usepackage{fancyhdr}
\fancyhf{}
\lhead{\textbf{FRCRCE}}
\rhead{\textbf{DEPARTMENT OF INFORMATION TECHNOLOGY}}

\rfoot{Compiled on \today\ at \currenttime \quad \textbf{\thepage}}
%\pagestyle{fancy}
\assigncourse{Course title: Big Data Analytics}
\assignterm{Course term: 2019-2020}
\assigncat{Practical}
%\assigntitle{}
%\renewcommand{\headrulewidth}{0pt}
%\renewcommand{\footrulewidth}{0pt}}
\lfoot{\textbf{Course title: Big Data Analytics}}

\fancypagestyle{plain}{}
\pagestyle{fancy}
\begin{document}
\sloppy
\fancypage{\doublebox}{}
\begin{flushleft}


    \begin{tabular}{ | p{4cm} | p{5cm} | p{3.5cm} | p{2cm} |}
    \hline

    \textbf{Name of the student:}& &\textbf{Roll No.} & \\ \hline
    \textbf{Practical Number:}& 5 & \textbf{Date of Practical:} & \\ \hline
	\textbf{Relevant CO's} & \multicolumn{3}{|p{10.5cm}|}{\begin{flushleft}
	\textbf{At the end of the course students will be able to use tools like hadoop and NoSQL to solve big data related problems.}
	\end{flushleft}}\\
    \hline
    \multicolumn{3}{|p{12.5cm}|}{\textbf{Sign here to indicate that you have read all the relevant material provided before attempting this practical}}& \textbf{Sign:}\\ \hline
    \end{tabular}
    \vspace{1cm}
        \textbf{Practical grading using Rubrics}
           \begin{tabular}{|p{2cm}|p{2cm}|p{2cm}|p{2cm}|p{2cm}|p{2cm}|}
                             \hline \textbf{Indicator} & \textbf{Very Poor} & \textbf{Poor} & \textbf{Average} & \textbf{Good} & \textbf{Excellent} \\ 
                             \hline \textbf{Timeline} (2) & More than a session late (0) & NA  & NA  & NA  & Early or on time (2) \\ 
                             \hline \textbf{Code design} (2) & N/A & Very poor code design with no comments and indentation(0.5) & Poor code design with very comments and indentation
                             (1) & Design with good coding standards (1.5) & Accurate design with better coding satndards (2) \\ 
                             \hline \textbf{Performance} (4) & Unable to
                             perform the
                             experiment
                             (0) & Able to
                             partially
                             perform the
                             experiment
                             (1)
                              & Able to
                              perform the
                              experiment
                              for certain use
                              cases (2) & Able to
                              perform the
                              experiment
                              considering
                              most of the
                              use cases (3)
                              & Able to
                              perform the
                              experiment
                              considering
                              all use cases
                              (4) \\ 
                             \hline \textbf{Postlab} (2) & No Execution(0) & N/A & Partially Executed (1) & N/A & Fully Executed (2) \\ 
                             \hline 
                             \end{tabular}
       % \vspace{-8cm}
        \begin{table}[h!]
        \centering
        \begin{tabular}{|c|c|}
                \hline \textbf{Total Marks (10)} & \textbf{Sign of instructor with date} \\ 
                \hline  &  \\ & \\
                \hline 
                \end{tabular} 
        \end{table}
        
    \pagebreak

	%\input{assignment}

\maketitle

%\thispagestyle{empty}
\hrule \vspace{0.2cm}
\textbf{Problem Statement: Perform CRUD operations in MongoDB}\hrule\vspace{0.2cm}
\textbf{Theory:Explain different CRUD Operations}\hrule\vspace{0.2cm}
\afterpage{\newpage~\newpage}\newpage
\textbf{Code:}\hrule
\vspace{0.5cm}
\textbf{\underline{Code of CRUD operations in MongoDB}}
\begin{enumerate}
\item Create documents for following data in collection called media.

\begin{table}[h]


\begin{tabular}{|p{1.5cm}|p{2cm}|p{2.0cm}|p{1.5cm}|p{6.5cm}|}
\hline \textbf{Type} & \textbf{Title} & \textbf{ISBN} & \textbf{Publisher} & \textbf{Author} \\ 
\hline Book & Def. guide & 978-1-482-0 & Apress & "Hows, David","Plugge,
Eelco","Membrey, Peter","Hawkins, Tim" \\ 
\hline Book & A text book on automata theory & 978-2-482-0 & Foundation books & "Nasir, S.F.B","Srimani,P.K" \\
\hline Book & MongoDB in Action & 978-3-482-0 & Manning Publication & "Banker, Kyle"\\
\hline Book & NoSQL for dummies & 978-4-482-0 & Wiley & "Fowler, Adam"\\
\hline Book & Big Data Analytics & 978-5-482-0 & Wiley & "Shankarmani, Radha"\\
\hline 
\end{tabular} 
\end{table}
\textbf{\underline{code for creating documents:}}
\begin{lstlisting}

\end{lstlisting}
\item Insert a document with type=CD, Artist=Nirvana, Title=Never Mind, Tracklist=[{Track:1,Title:Smells Like Teen Spirit,length:5:02},{Track:2,Title:In Bloom,length:4:15}] in same collection named media.\\
\textbf{\underline{code for creating document:}}
\begin{lstlisting}

\end{lstlisting}
\item Find all documents in collection named media.

\textbf{\underline{Query Code:}}
\begin{lstlisting}

\end{lstlisting}
\item Find documents where publisher is Wiley
\\
\textbf{\underline{Query Code:}}
\begin{lstlisting}

\end{lstlisting}
\item Find titles of CDs whose artist is Nirvana.
\\
\textbf{\underline{Query Code:}}
\begin{lstlisting}

\end{lstlisting}
\item Find all documents sorted in descending order.
\\
\textbf{\underline{Query Code:}}
\begin{lstlisting}

\end{lstlisting}
\item Find only 3 documents of book type.
\\
\textbf{\underline{Query Code:}}
\begin{lstlisting}

\end{lstlisting}
\item Find last 3 documents from collection named media.
\\
\textbf{\underline{Query Code:}}
\begin{lstlisting}

\end{lstlisting}
\end{enumerate}

\newpage
\textbf{PostLab:}\hrule
Compute customerwise total amount on given dataset using map reduce for the customers with status as "A" and store this result in a document called Order\_total.\\
\textbf{\underline{code for mapreduce function of MongoDB}}
\begin{lstlisting}

	
\end{lstlisting}
Calculate number of times the site has been visited using mapreduce in MongoDB on given dataset.
\textbf{\underline{Code for the question}}
\begin{lstlisting}

	
\end{lstlisting}
\newpage

\end{flushleft}
\end{document}

