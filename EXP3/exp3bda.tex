\documentclass[11pt,article]{memoir}
\usepackage{multirow}
\usepackage{agd-assignment}
\usepackage{fancybox}
\usepackage{afterpage}
\usepackage{graphicx}
\usepackage{standalone}
\usepackage[yyyymmdd,hhmmss]{datetime}


\usepackage{listings}
\lstset{
    frame=single,
    breaklines=true,
    postbreak=\raisebox{0ex}[0ex][0ex]{\ensuremath{\color{red}\hookrightarrow\space}}
}
\let\footruleskip\undefined
\usepackage{fancyhdr}
\fancyhf{}
\lhead{\textbf{FRCRCE}}
\rhead{\textbf{DEPARTMENT OF INFORMATION TECHNOLOGY}}

\rfoot{Compiled on \today\ at \currenttime \quad \textbf{\thepage}}
%\pagestyle{fancy}
\assigncourse{Course title: Big Data Analytics}
\assignterm{Course term: 2016-2017}
\assigncat{Practical}
%\assigntitle{}
%\renewcommand{\headrulewidth}{0pt}
%\renewcommand{\footrulewidth}{0pt}}
\lfoot{\textbf{Course title: Big Data Analytics}}

\fancypagestyle{plain}{}
\pagestyle{fancy}
\begin{document}
\sloppy
\fancypage{\doublebox}{}
\begin{flushleft}


    \begin{tabular}{ | p{4cm} | p{5cm} | p{3.5cm} | p{2cm} |}
    \hline

    \textbf{Name of the student:}& &\textbf{Roll No.} & \\ \hline
    \textbf{Practical Number:}& 3 & \textbf{Date of Practical:} & \\ \hline
	\textbf{Relevant CO's} & \multicolumn{3}{|p{10.5cm}|}{\begin{flushleft}
	\textbf{At the end of the course students will be able to use tools like hadoop and NoSQL to solve big data related problems.}
	\end{flushleft}}\\
    \hline
    \multicolumn{3}{|p{12.5cm}|}{\textbf{Sign here to indicate that you have read all the relevant material provided before attempting this practical}}& \textbf{Sign:}\\ \hline
    \end{tabular}
    \vspace{1cm}
        \textbf{Practical grading using Rubrics}
           \begin{tabular}{|p{2cm}|p{2cm}|p{2cm}|p{2cm}|p{2cm}|p{2cm}|}
           \hline \textbf{Indicator} & \textbf{Very Poor} & \textbf{Poor} & \textbf{Average} & \textbf{Good} & \textbf{Excellent} \\ 
           \hline \textbf{Timeline} (2) & Practical not submitted (0) & More than two session late (0.5) & Two sessions late (1) & One session late (1.5) & Early or on time (2) \\ 
           \hline \textbf{Code design} (3) & N/A & Very poor code design(0) & poor design (1) & design with good coding standards (2) & Accurate Design with better coding standards(3) \\ 
           \hline \textbf{Execution} (3) & N/A & Very less execution (0)
            & little execution.(1) & Major execution(2)
            & Entire code execution (3) \\ 
           \hline \textbf{Postlab} (2) & No Execution(0) & N/A& Partially Executed (1) & N/A & Fully Executed (2) \\ 
           \hline 
           \end{tabular}
        \vspace{-8cm}
        \begin{table}[h!]
        \centering
        \begin{tabular}{|c|c|}
                \hline \textbf{Total Marks (10)} & \textbf{Sign of instructor} \\ 
                \hline  &  \\ 
                \hline 
                \end{tabular} 
        \end{table}
        
    \pagebreak

	%\input{assignment}

\maketitle

%\thispagestyle{empty}
\hrule \vspace{0.2cm}
\textbf{Problem Statement: Find year-wise maximum temperature from the given dataset using map-reduce.}\hrule\vspace{0.2cm}
\textbf{Theory:Explain the working of finding maximum temperature using map reduce with small example and diagrams}\hrule\vspace{0.2cm}
\afterpage{\newpage~\newpage}\newpage
\textbf{Code:}\hrule
\vspace{0.5cm}
\textbf{\underline{code for mapper:}}

% Write code of mapper here
\begin{lstlisting}[language=java]

	
\end{lstlisting}
\textbf{\underline{Code for Reducer:}}
\begin{lstlisting}[language=java]





\end{lstlisting}
% Write code for reducer here
\textbf{\underline{Code for Driver Class:}}
\begin{lstlisting}[language=java]





\end{lstlisting}
\newpage
\textbf{PostLab:}\hrule
Can we use combiner function in the above code? If yes, write the above code with combiner function.
\textbf{\underline{Code for postlab question}}                        
                          
\newpage

\end{flushleft}
\end{document}

\end{flushleft}
\end{document}